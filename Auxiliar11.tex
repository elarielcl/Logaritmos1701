\documentclass[dcc,uchile]{fcfmcourse}
\usepackage{teoria}
\usepackage[utf8x]{inputenc}
\usepackage{amsmath}
\usepackage{amsfonts,setspace}
\usepackage{listings}
\usepackage{hyperref}
\usepackage{color}
\usepackage{soul}
\usepackage{emojis}

\usepackage[lined,boxed,commentsnumbered]{algorithm2e}
\SetKwInput{KwIn}{Input}
\renewcommand{\algorithmcfname}{Algoritmo}


\definecolor{pblue}{rgb}{0.13,0.13,1}
\definecolor{pgreen}{rgb}{0,0.5,0}
\definecolor{porange}{rgb}{0.9,0.5,0}
\definecolor{pgrey}{rgb}{0.46,0.45,0.48}

\lstset{language=Java,
  showspaces=false,
  showtabs=false,
  breaklines=true,
  showstringspaces=false,
  breakatwhitespace=true,
  commentstyle=\color{porange},
  keywordstyle=\color{pblue},
  stringstyle=\color{pgreen},
  basicstyle=\ttfamily,
  moredelim=[il][\textcolor{pgrey}]{$ $},
  moredelim=[is][\textcolor{pgrey}]{\%\%}{\%\%}
}

\newenvironment{codebox} {\small \ttfamily \obeylines \begingroup \setstretch{-2.4}} {\endgroup}
 
\title{Auxiliar 11 - ``Algoritmos Paralelos"}
\course[CC4102]{Diseño y Análisis de Algoritmos}
\professor{Pablo Barceló}
\assistant{\textst{Manuel} Ariel Cáceres Reyes}


\begin{document}
\maketitle
\begin{center}
16 de Junio del 2017
\end{center}
\vspace{-1ex}

%%ORDENAR PUNTOS EN EL PLANO CON DISTRIBUCION UNIFORME
%%ORDENAR VALORES EN [0, n^k - 1]
%%ORDENAR SEGUN DISTANCIA AL CENTRO, PUNTOS DEL CIRCULO UNITARIO
%%vEM (?)
%%RANK EN TIEMPO CTE Y ESPACIO o(n)
%%EXPRESIONES REGULARES EN AUTOMATAS! (COMODIN)
%%KNUTH MORRIS PRATT
%%OPERACIONES EN SUFFIX TREES!

\begin{problems}
%Bienvenida
\problem {\large\underline{\textbf{Copias}}}\\
Se desea, dado un cierto valor $x$, generar $n$ copias de este.
\begin{enumerate}[a)]
    \item Diseñe un algoritmo CREW de tiempo $\mathcal{O}(1)$ y eficiencia $\Theta(1)$
    \item Diseñe un algoritmo EREW de tiempo $\mathcal{O}(\log n)$ que use $n$ procesadores
    \item Mejórelo para obtener eficiencia $\Theta(1)$
\end{enumerate}

\problem {\large\underline{\textbf{Multiplicar Matrices}}}\\
Considere el problema de multiplicar dos matrices de $n\times n$. Considere que $T(n) = \mathcal{O}(n^3)$, es decir, ignoramos algoritmos mejores al estándar.
\begin{enumerate}[a)]
    \item Proponga un algoritmo de tiempo $\mathcal{O}(n)$ usando $n^2$ procesadores. Calcule $T(n, p), W(n), E(n, p)$. ¿Qué modelo PRAM usa?
    \item Proponga un algoritmo de tiempo $\mathcal{O}(\log n)$ usando $n^3$ procesadores. Calcule $T(n, p), W(n), E(n, p)$
    \item Mejore la eficiencia del algoritmo anterior para que sea $\Theta(1)$. ¿Qué tiempo obtiene y cuántos procesadores utiliza?
\end{enumerate}
\problem {\large\underline{\textbf{Select(A, 1)}}}\\
Sea $A$ un arreglo de bits de largo $n$. Diseñe un algoritmo CRCW que tome tiempo $\mathcal{O}(1)$ utilizando $\mathcal{O}(n)$ procesadores para encontrar Select(A, 1), el primer elemento $k$ tal que $A[k] = 1$.
\end{problems}

\newpage
\begin{problems}
\problem
\begin{enumerate}[a)]
\item
Los $n$ procesadores miran a $x$ y lo copian en $B[p]$ en $1$ paso paralelo.
Con esto tenemos que:
\begin{align*}
    T(n,n) &= 1\\
	T(n) &= n\\
	E(n,n) &= \frac{n}{n\cdot 1} = \mathcal{O}(1)
\end{align*}
\item 
Lo anterior es CREW, para hacerlo EREW duplicamos los $x$'s en cada paso usando el doble de procesadores que en el paso anterior, hasta $\frac{n}{2}$.\\
\begin{algorithm}[H]

\SetAlgoLined
\For{$i \gets 0 \ldots (log(n) - 1)$}{
    \If {$p \le 2^i$}{
        $B[2^i+p] = B[p]$
    }
}
\end{algorithm}
Notemos ahora que son $\log{n}$ pasos y se ocupan $\frac{n}{2}$ procesadores, por lo que:
\begin{align*}
T(n,n/2) &= \log{n}\\
E(n,n/2) &= \frac{n}{n/2\cdot \log{n}} = \mathcal{O}(1/\log{n})
\end{align*}
\item
>Primero veamos que $W(n)=n$ pues se siguen copiando la misma cantidad de elementos.\\
Ahora, por lema de Brent tenemos que existe un algotimos EREW tal que:
\begin{align*}
T\left(n,\frac{n}{\log{n}}\right) &= \mathcal{O}(\log{n})\\
E\left(n,\frac{n}{\log{n}}\right) &= \frac{n}{\frac{n}{\log{n}}\cdot \mathcal{O}(\log{n})} = \mathcal{O}(1)
\end{align*}
\end{enumerate}
\problem
\begin{enumerate}[a)]
\item
Si tenemos $n^2$ procesadores a cada uno le asignamos el cálculo de $C_{i,j}$ y por lo tanto, podemos hacer la multiplicación en $\mathcal{O}(n)$ pasos paralelos (lo que tarda un producto punto).\\ Luego:
\begin{align*}
T(n,n) &= \mathcal{O}(n)\\
E(n,n) &= \frac{n^3}{n^2\cdot \mathcal{O}(n)} = \mathcal{O}(1)
\end{align*}
Notar que este es un algoritmo CREW pues hay procesadores que miran a la misma fila y columna.
\item
Con $n^3$ procesadores:
\begin{itemize}
    \item En un paso paralelo calculamos $A_{i,k}\cdot B_{k,j}, \forall i,j,k$ y lo ponemos en $C_{i,k,j}$.
    \item Queremos calcular $C_{i,j} = \sum_{k\in [n]} C_{i,k,j}$ en $\mathcal{O}(\log{n})$ pasos.
\end{itemize}
Para realizar esto asignaremos $n$ procesadores a cada par $(i,j)$ quienes se encargarán de hacervla suma de estos $n$ elementos. Para alcanzar el $\mathcal{O}(\log{n})$ aplicaremos la misma idea de ``torneo'' utilizada para el problema de encontrar el máximo en un arreglo, según el siguiente código:\\
\begin{algorithm}[H]

\SetAlgoLined
\For{$d \gets 0 \ldots (log(n) - 1)$}{
    \If {$(p-1) \% 2^{d+1} = 0$}{
        $C[i][p][j] = C[i][p][j] + C[i][p+2^d][j]$
    }
}
\end{algorithm}
Con esto tenemos:
\begin{align*}
T(n,n^3) &= \log{n}\\
E(n,n^3) &= \frac{n^3}{n^3\cdot \log{n}} = \mathcal{O}(1/\log{n})
\end{align*}
\item
Veamos que $W(n) = n^3$, pues las multiplicaciones son $n^3$ y después hacemos $n^2$ sumas, cada una considerando $n$ sumandos.\\
Lamentablemente, aunque las sumas son EREW, el cálculo de los $C_{i,k,j}$ no lo es. Para poder aplicar el lema de Brent solucionaremos este último problema.\\
Para el cálculo de un $C_{i,k,j} = A_{i,k}\cdot B_{k,j}$, tendremos $n$ procesadores revisando una misma casilla de $A$ y $n$ revisando una misma casilla de $B$, por lo tanto haciendo $n$ copias de cada una de las matrices tenemos el problema solucionado.\\
Como vimos en el P1, podemos hacer $n$ copias de un elemento en $\log{n}$ pasos paralelos con $n$ procesadores, de este modo, podemos copiar una $A$ en $\log{n}$ pasos paralelos con $n^3$ procesadores, lo que no aumenta la complejidad (paralela) del problema
\end{enumerate}
\problem
Dividimos el bitmap en bloques de tamaño $\sqrt{n}$, creando un bitmap
\begin{align*}
    B[i] = \begin{cases} 
      0 & si\ A[(i-1)\sqrt{n}, i\sqrt{n}] = 0s \\
      1 & si\ \exists 1 \in A[(i-1)\sqrt{n}, i\sqrt{n}] 
   \end{cases}
\end{align*}
\begin{itemize}
    \item $B$ puede ser llenado en $1$ paso paralelo (los $n$ procesadores miran cada uno una casilla de $A$ y si ven un $1$ ponen un $1$ en la casilla correspondiente de $B$.
    \item En otro paso paralelo podemos identificar el primer bloque que contiene un $1$, sea $i^*$ este bloque.
    \item Finalmente, obtenemos el primer uno del bloque $i^*$, llamándole $j^*$ con lo que el resultado sería $(i^*-1)\sqrt{n}+j^*$.
\end{itemize}
Para hacer esto último aplicamos un algoritmo bárbaro, que tiene un procesador mirando cada par de elementos del bloque (como el bloque es de tamaño $\sqrt{n}$ existen $\mathcal{O}(\sqrt{n}^2) = \mathcal{O}(n)$ pares) y si ambos elementos del par tienen un $1$ escribe un $0$ en el elemento de más a la derecha, de este modo, al finalizar, el único que tendrá un $1$ en el bloque será $j^*$.
\end{problems}
\end{document}

 