\documentclass[dcc,uchile]{fcfmcourse}
\usepackage{teoria}
\usepackage[utf8x]{inputenc}
\usepackage{amsmath}
\usepackage{amsfonts,setspace}
\usepackage{listings}
\usepackage{hyperref}
\usepackage{color}
\usepackage{soul}
\usepackage{emojis}

\usepackage{algorithmic}


\definecolor{pblue}{rgb}{0.13,0.13,1}
\definecolor{pgreen}{rgb}{0,0.5,0}
\definecolor{porange}{rgb}{0.9,0.5,0}
\definecolor{pgrey}{rgb}{0.46,0.45,0.48}

\lstset{language=Java,
  showspaces=false,
  showtabs=false,
  breaklines=true,
  showstringspaces=false,
  breakatwhitespace=true,
  commentstyle=\color{porange},
  keywordstyle=\color{pblue},
  stringstyle=\color{pgreen},
  basicstyle=\ttfamily,
  moredelim=[il][\textcolor{pgrey}]{$ $},
  moredelim=[is][\textcolor{pgrey}]{\%\%}{\%\%}
}

\newenvironment{codebox} {\small \ttfamily \obeylines \begingroup \setstretch{-2.4}} {\endgroup}
 
\title{Auxiliar 7 - ``Dominios Discretos"}
\course[CC4102]{Diseño y Análisis de Algoritmos}
\professor{Pablo Barceló}
\assistant{\textst{Manuel} Ariel Cáceres Reyes}


\begin{document}
\maketitle
\begin{center}
26 de Mayo del 2017
\end{center}
\vspace{-1ex}

%%ORDENAR PUNTOS EN EL PLANO CON DISTRIBUCION UNIFORME
%%ORDENAR VALORES EN [0, n^k - 1]
%%ORDENAR SEGUN DISTANCIA AL CENTRO, PUNTOS DEL CIRCULO UNITARIO
%%vEM (?)
%%RANK EN TIEMPO CTE Y ESPACIO o(n)
%%EXPRESIONES REGULARES EN AUTOMATAS! (COMODIN)
%%KNUTH MORRIS PRATT
%%OPERACIONES EN SUFFIX TREES!

\begin{problems}
%Bienvenida
\problem {\large\underline{\textbf{Rank}}}\\
Sea $B$ una secuencia de bits de largo $n$. Se define $RANK(B, i)$ como el número de bits en $1$ en $B[1,i]$, es decir:
\begin{align*}
    RANK(B, i) &= \sum_{0<j\le i} B[j],\ 1\le i \le n
\end{align*}
\begin{enumerate}[a)]
    \item Construya una estructura que permita calcular $RANK(B,i)$ en tiempo constante y utilice $2n + o(n)$ \textbf{bits} de espacio.
    \item Resuelva el mismo problema, esta vez utilizando $o(n)$ \textbf{bits} de espacio.
\end{enumerate}

\problem {\large\underline{\textbf{Árbol de Sufijos}}}\\
Muestre lo siguiente sobre el Árbol de Sufijos $ST$ de un  \texttt{String} $T$ de largo $n$.
\begin{enumerate}[a)]
    \item El espacio utilizado por $ST$ es $\mathcal{O}(n)$.
    \item ¿Como conocer el número de veces que un \texttt{String} $P$ aparece como \texttt{substring} de $T$? ¿Cuánto cuesta? ¿Cómo lo puedo mejorar?
    \item ¿Cómo saber cual es el \texttt{substring} más largo de $T$  que aparece $k$ veces? ¿Cual es el costo con $ST$?
    \item ¿Cómo puedo construir el \texttt{Arreglo de Sufijos} de $T$? ¿Cual es el costo?
\end{enumerate}


\problem {\large\underline{\textbf{Ordenar}}}\\
Ordene $n$ números enteros en el rango $[1, n^2]$ en $\mathcal{O}(n)$. Generalice su resultado para dominios  de la forma $[1, n^k]$
\end{problems}

\newpage
\begin{center}
{\huge \underline{Soluciones}}
\end{center}
\begin{problems}
\item 
\begin{enumerate}[a)]
\item Una primera idea consiste en tener un arreglo de $n$ elementos que en la posición $i$ contenga el valor de $RANK(B,i)$, sin embargo, como el $RANK$ puede llegar hasta $n$ necesito $\mathcal{O}(\log n)$ bits para representar uno de estos valores y en total $\mathcal{O}(n\log n)$. \crying\\

\idea\\
Dividir $B$ en partes de tamaño $\frac{\lceil \log n \rceil}{2}$ y guardar el $RANK$ de los últimos elementos de cada una de las partes. Esto toma $\frac{n}{\frac{\lceil \log n \rceil}{2}}\log n = \frac{2n}{\lceil \log n \rceil}\log n \le 2n$ bits de espacio.\\
Ahora que tengo el $RANK$ cada $\frac{\lceil \log n \rceil}{2}$ solo debo identificar el $RANK$ más cercano y luego calcular el resto en tiempo $\mathcal{O}\left(\frac{\lceil \log n \rceil}{2}\right)$. \\
\idea\\
Para reducir la complejidad a $\mathcal{O}(1)$ precalcularemos todos los $RANK$ de todas las secuencias de largo $\frac{\lceil \log n \rceil}{2}$. Es decir, tendremos en una tabla $T[w,r] = RANK(w,r)$ que ocupará espacio $2^{\frac{\lceil \log n \rceil}{2}} \cdot \frac{\lceil \log n \rceil}{2} \cdot \log n = \mathcal{O}(\sqrt{n})\cdot \mathcal{O}(\log n) \cdot \log n = \mathcal{O}(\sqrt{n} \log^2 n) \in o(n)$ bits.\wedidit  
\item El problema con la solución anterior es que el arreglo que guarda los $RANK$ de las partes ocupa $\mathcal{O}(n)$ bits de espacio.\\
\idea\\
Aumentaremos el tamaño de las partes a $\frac{\log^2 n}{2}$, de este modo solo necesitamos $\frac{n}{\frac{\log^2 n }{2}}\log n = \frac{2n}{\log n } = o(n)$ bits de espacio.\\
Lamentablemente ya no podemos guardar todos los $RANK$ de todas las secuencias de tamaño $\frac{\log^2 n}{2}$ en espacio $o(n)$. \crying\\
\idea\\
Dividimos cada parte en $\log n$ subpartes de tamaño $\frac{\log n}{2}$ cada una y guardamos el $RANK$ del final de cada una, pero relativo a la parte en la cual se encuentra esa subparte. De este modo necesitamos $\log n \cdot \frac{n}{\frac{\log^2 n}{2}} \cdot \log \left(\frac{\log^2 n}{2}\right) \le \frac{4n\log \log n}{\log n} = o(n)$. \wedidit
\end{enumerate}
\item 
\begin{enumerate}[a)]
\item Primero veamos que el Árbol de Sufijos tiene $\mathcal{O}(n)$ nodos. Esto es así pues todos los nodos, excepto quizás la raíz, tienen al menos $2$ hijos. Luego como hay $n+1$ hojas (una por cada sufijo y una por $\$$) el árbol tiene a lo más $n+1$ nodos internos\footnote{Esto puede ser demostrado por inducción} y por lo tanto a lo más $2n + 2 = \mathcal{O}(n)$ nodos. Finalmente, para no aumentar el espacio, el árbol no guardará los substrings en sus aristas (ni en sus nodos), sino que los índices correspondientes en el texto original.
\item Un \texttt{String} $P$ es un substring de $T$ si y solo si $P$ es un prefijo de un sufijo de $T$. Luego para saber el número de veces que $P$ aparece como substring de $T$ podemos bajar por el Árbol de Sufijos de $T$ con $P$ en $\mathcal{O}(|P|)$ y luego contar cuantas hojas hay a partir del nodo que se llegó en $\mathcal{O}(\# substrings)$. Para mejorar esto podemos almacenar el número de hojas bajo un nodo en cada nodo del árbol, reduciendo el costo de la operación anterior a $\mathcal{O}(|P|)$.
\item  Lo pedido es equivalente a encontrar el nodo más profundo (en términos de número de letras) que tiene al menos $k$ hojas, lo que se puede hacer recorriendo el árbol en $\mathcal{O}(|T|)$.\perfect
\item Si hacemos un recorrido inorden en el Árbol de Sufijos llegamos a las hojas en orden lexicográfico, como las hojas representan sufijos del texto estamos recorriendo los sufijos del texto en orden lexicográfico a partir de lo cual se puede obtener el Arreglo de Sufijos correspondiente.
\end{enumerate}
\item Si consideramos los números que estamos ordenando en base $b$, Counting-Sort toma $\mathcal{O}(n+b)$, luego si $k$ es el número más grande a considerar, Radix-Sort debe hacer $\log_{b} k$ Counting-Sorts y por lo tanto toma $\mathcal{O}((n+b)\log_{b} k)$.\\

Si escogemos $n$ como la base tenemos que:
\begin{itemize}
    \item Ordenar enteros en el rango $[1, n^2]$ toma $\mathcal{O}((n+n)\log_{n} n^2) = \mathcal{O}(4n) = \mathcal{O}(n)$.
    \item Ordenar enteros en el rango $[1, n^k]$ toma $\mathcal{O}((n+n)\log_{n} n^k) = \mathcal{O}(2kn)$.
\end{itemize}
\end{problems}

\end{document}

 